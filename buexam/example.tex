% Note this requires the (standard) exam package
\documentclass[a4paper,12pt,addpoints]{exam} 

% Your distro may not have these packages. On ubuntu, you need texlive-science. On windows, miktex should find them for you.
\usepackage{amsmath}
\usepackage{algorithm}
\usepackage{algorithmic}

%This is the package which you need to include
\usepackage{buexam}
% Specify the course name here
\renewcommand{\coursetitle}{Software Development for Animation, Games and Effects}
% Give an acronym or short version of the course name here
\renewcommand{\coursetitleshort}{SDAGE}
% The Unit title of the exam (append RESIT to the end if necessary)
\renewcommand{\unittitle}{Zombies in Popular Media and Mathematics}
% A shortened / abbreviated name of the unit
\renewcommand{\unittitleshort}{ZPMM}
% The unit reference (might be the same as short name)
\renewcommand{\unitref}{ZPMM}
% The level of the examination (C,I,H etc)
\renewcommand{\unitlevel}{I}
% The exam reference (get this from the school
\renewcommand{\examref}{ZZZ1234Z}
% The date of the exam
\renewcommand{\examdate}{10/06/2013}
% The time of the exam
\renewcommand{\examtime}{12h00}
% The time that this exam was created - removing the line defaults to todays date (date of compile)
%\renewcommand{\exampublishdate}{\today}
% Unit leader of the course
\renewcommand{\unitleadername}{Richard Southern}
% Unit leader extension
\renewcommand{\unitleaderext}{61877}
% Assistant name
\renewcommand{\assistantname}{Ben Ellis}
% Assitant extension
\renewcommand{\assistantext}{65745}

% Insert series table contents here: first column is programme name, second column is corresponding programme code
% Note that this is for all programmes that share this unit (and exam). Currently this is supported up to C (i.e. 3 different
% programmes) - it is easily ammended in the stylesheet.
\renewcommand{\stagetitleA}{BA (Hons) Communication and Media – Level I}
\renewcommand{\stagerefA}{BACOMMF/I}
\renewcommand{\stagetitleB}{BA (Hons) English – Level I}
\renewcommand{\stagerefB}{BAEF/I}
                  
% There are the specific instructions for candidates
\renewcommand{\instructionstocandidates}{
\begin{itemize}
\item There are \textbf{THREE} questions
\item Answer \textbf{ALL} questions
\item All questions carry equal marks
\item \textbf{NO} calculators, laptops or any other electronic equipment are permitted
\item \textbf{NO} additional materials may be used
\end{itemize}
}

% This option should be commented out for the final paper (don't forget!)
\printanswers

\begin{document}

% Note that this command will create ``Section B\\Answer All Questions'' - you have to use this, as it is in an itemized environment.
\sectioninstructions{Answer ALL Questions}

% Questions can consist of multiple parts. Note that the sum of the parts must make the total (in this case, [15]).
\titledquestion{Zombies in films}[15]
Consider Zombies in films. 
\begin{parts}\addtolength{\itemsep}{-0.5\baselineskip}
% Simple  question example
\part[5] Name some films with Zombies \droppoints
\begin{solution}
\begin{itemize}
\item Pride and Prejudice... with Zombies!
\item Abraham Lincoln... Zombie Hunter!
\item etc.
\end{itemize}
\end{solution}

% Maths example
\part[10] State the average number of characters eaten by a single Zombie in "`28 Days Later"'. Use the following formula:
\begin{align}
\bar{x} &= \frac{1}{n}\cdot \sum_{i=1}^{n} x_i \nonumber
\end{align}\droppoints
\begin{solution}
At least 3. Zombies are awesome.
\end{solution}
\end{parts}

% A newpage is required between sections
\newpage

% Note that this command will create ``Section B\\Answer Only ODD numbered questions'' - you have to use this, as it is in an itemized environment.
\sectioninstructions{Answer Only ODD numbered questions}

\titledquestion{Zombie Programming}[10]
Consider the following code segment:
\begin{lstlisting}
#include <iostream>
using namespace std;

int main ()
{
  cout << "Hello World!";
  return 0;
}
\end{lstlisting}
In your own words, explain:
\begin{parts}
%Example of code inclusion
\part[10] Was this code produced by a Zombie? If so, why?\droppoints
\begin{solution}
Clearly, the lameness indicates that it is Zombie generated code (ZGC).
\end{solution}

\part[5] How would you improve on the above code? \droppoints
\begin{solution}
Don't use the std namespace! Better to give your fingers exercise and type it every time.
\end{solution}
\end{parts}
\newpage

\titledquestion{Zombie Algorithms}[15]
\begin{parts}

% Example of an algorithm
 \part[10] Consider the following algorithm:

\begin{algorithmic}
\REQUIRE $n \geq 0$
\ENSURE $y = x^n$
\STATE $y \leftarrow 1$
\STATE $X \leftarrow x$
\STATE $N \leftarrow n$
\WHILE{$N \neq 0$}
\IF{$N$ is even}
\STATE $X \leftarrow X \times X$
\STATE $N \leftarrow N / 2$
\ELSE[$N$ is odd]
\STATE $y \leftarrow y \times X$
\STATE $N \leftarrow N - 1$
\ENDIF
\ENDWHILE
\end{algorithmic}

Describe in your own words the thought processes required for a Zombie to generate this algorithm. How can the promise of fresh brains serve as a motivators? \droppoints
\begin{solution}
Blah blah blah.
\end{solution}

% Example of image inclusion
\part[5] Admire our institution's mighty logo:
\begin{center}
\includegraphics[width=0.1\textwidth]{bulogo}
\end{center}
Draw this in a manner similar to a Zombie. Coloured pens are provided.\droppoints
\begin{solution}
Blah blah blah.
\end{solution}

\end{parts}


% The postamble is needed as part of the uni style. (this cannot be automatic as AtEndOfDocument adds a \newpage)
\afterlastquestion

\end{document}


